\documentclass[11pt,twocolumn]{article}
\usepackage{fullpage}
\usepackage{amsmath}
\usepackage{spverbatim}
\usepackage{url}

\title{Journal Characteristics}
\author{Ankita Upadhyay }
\date{October 2018}

\begin{document}
\maketitle
\section{ACM (Journal 1)}
\begin{itemize}
  \item{\bf{Publisher}}: ACM New York, NY, USA, 
  \item {\bf{Name}}: ACM Transactions on Sensor Networks (TOSN)
  \item{\bf{url}}: \url{https://tosn.acm.org/}
  \item {\bf{A paragraph describing the scope of the journal}}: The ACM Transactions on Sensor Networks (TOSN) includes research regarding sensor networks which consists of actuator, distributed, and wireless/wireline networks. TOSN also has publications on the topic of network protocols, embedded systems, signal processing, distributed algorithms, and information management. TOSN encompasses applications of sensor networks including data storage, query processing, security/privacy configurations, and operating systems - to name a few. The scope is quite broad since TOSN publications can range from database management of individual components to large-scale system architecture and modeling.
  \item{\bf{Editor-in-Chief}}: Yunhao Liu
  \item{\bf{Editorial Board Size}}: 40
  \item{\bf{How frequently the  journal is}}
  {\bf{published}}: The journal is published four times a year.
  \item{\bf{A brief summary of submission instructions for the authors}}: Authors are required to render their manuscripts in accordance with the ACM manuscript guidelines which can be prepared using either LaTeX or MS Word. Manuscripts must be submitted electronically and in PDF format. The paper must also be copy edited and follow the guidelines stated in the {\textit{Chicago Manual of Style}} and the {\textit{Merriam Webster Dictionary}}. TOSN accepts unpublished original technical contributions as well as previously submitted contributions that have been modified and altered significantly. Any publication that has already been submitted must be noted and there must exist a cover letter that states the differences of the original paper and the one that was submitted. Authors are not allowed to resubmit papers that TOSN already rejected.
  \item{\bf{Whether or not the journal has page}}
  {\bf{limits for submitted articles}}: 
  30-40 pages is the page limit for submitted articles.
  \item – {\bf{Whether or not the journal has a }}
  {\bf{double-blind reviewing policy}}:
  Yes, ACM TOSN has adopted a double-blind reviewing policy. Anonymity on both ends is vital in order to prevent bias in reviewing.
  \item{\bf{Whether or not the journal has an open access policy; if so, briefly describe it:}}
  ACM TOSN has a hybrid open access policy where viewers can obtain free online access by paying a fee upfront. Authors can either manage their own publication rights or allow ACM to do so. Authors also have patent/trademarks rights that the ACM is not granted. 
  \item{\bf{At least one additional piece of}}
  {\bf{information about the journal that}}
  {\bf{(you find of interest e.g., it’s impact factor }}
  {\bf{compared to others in the same field)}}:
  The statistics in the bibliometrics were of my interest. There were 1834 downloads in six weeks, 17676 downloads in a year, 323787 cumulative downloads, and 4089 citations to date.
\end{itemize}


\maketitle

\section{Elsevier (Journal 2)}
\begin{itemize}
  \item{\bf{Publisher}}: Elsevier, Netherlands, Amsterdam 
  \item {\bf{Name}}: Journal of Process Control
  \item{\bf{url}}: \url{https://www.journals.elsevier.com/journal-of-process-control}
  \item {\bf{A paragraph describing the scope of the journal}}: The scope of this journal is broad since it encompasses engineering principles, control theory, and operations research on the topic of process control. This journal includes several applications such as data analytics, bio-medical engineering, nano-technology, and energy processes. This journal contains publications that delve into design methods and process control systems in the aforementioned fields as well. However, topics regarding machinery and robotics are not necessarily acceptable unless there exists some connection to process control. 
  \item{\bf{Editor-in-Chief}}: Denis Dochain
  \item{\bf{Editorial Board Size}}: 34
  \item{\bf{How frequently the  journal is}}
  {\bf{published}}: Elsevier publishes around one to two articles each month.
  \item{\bf{A brief summary of submission instructions for the authors}}: The author can only submit either a research paper or a review paper that contains original research of a renowned theoretical subject. The author can only publish a paper that has been previously published as long as they address the original contribution and differences among that and the original conference paper. All manuscripts must comply with the journal policies/requirements. The manuscript should include relevant captions, tables, proper citations, and keywords. The author must use LaTeX or Word to render their manuscript which the system will convert into a PDF file. The author must include their contact details such as their postal address and e-mail address upon submitting their contribution.  
  \item{\bf{Whether or not the journal has page}}
  {\bf{limits for submitted articles}}: 
  The research papers submitted in this particular journal can only be about 7500 words, which is 15 pages single spaced.
  \item{\bf{Whether or not the journal has a }}
  {\bf{double-blind reviewing policy}}:
  Yes, Elsevier has a double-blind reviewing policy.
  \item{\bf{Whether or not the journal has an open access policy; if so, briefly describe it:}}
  In terms of publishing their research, Elsevier allows authors to choose from either a Gold Open Access plan which entails free access of articles to the public and subscribers with a fee or a subscription based plan that does not include any publication fee.  
  \item{\bf{At least one additional piece of}}
  {\bf{information about the journal that}}
  {\bf{(you find of interest e.g., it’s impact factor }}
  {\bf{compared to others in the same field)}}:
  This cite score for this journal is 2.69. This is the ratio of citations to date (878) over documents to date (326). 2018 is not yet over but in terms of the current statistics, there has been a decrease from 3.85 (2017) to 2.69 (2018).   
  
 
 \section{Three Recently Published Journals From Each of the Selected Journals}

\subsection{ACM Transactions on Sensor Networks (TOSN)}

{{\bf{Using mobile phones to determine transportation modes}}
\begin{itemize}
\item This title is not too broad, yet it could be more narrow. The reader can get a good sense of what the article might talk about but also wonder what the scope might be. Before reading the article I predicted that there might be details about GPS and navigation. While this was true, the article delved deep into the specifications and parametrics of data collection. Overall, I would conclude that this title is appropriate but could be more specific in order to let the user know what to expect when it comes to certain transportation modes. My suggestion would be to specify what transportation modes in particular this article focuses on.  

\end{itemize}

{\bf{
Opportunistic Spectrum Allocation for Interference Mitigation Amongst Coexisting Wireless Body Area Networks}}
\begin{itemize}
\item This title is the epitome of a viable title for a paper on sensor networks. It is slightly on the longer side but gives the reader an in-depth sense of the specifics of wireless body area networks. The "opportunistic spectrum allocation" might take some time to digest, but upon reading the article, it is clear that the "spectrum" is indicative of the practicality of certain networks that impact signal blockage. My suggestion would be to remove "opportunistic" and "spectrum" and replace it with words that sound more technical in terms of what a computer science research article embodies.
\end{itemize}


{\bf{ATPC: Adaptive Transmission Power Control for Wireless Sensor Networks}}
\begin{itemize}
\item This title is highly effective as the scope is not too broad nor too lengthy unlike the former two, respectively. The reader knows what to expect when it comes to research about wireless networks and can make educational guesses as to what the term "adaptive transmission power" might entail. This title captures as well as keeps readers' attention since it gives them a good sense of what to expect at any given point of the article. 
\end{itemize}

\subsection{Elsevier (Journal of Process Control)}

{\bf{Efficient energy management of CO2 capture plant using control-based optimization approach under plant and market uncertainties}}
\begin{itemize}
    \item This article title is effective to a slight extent as it is too wordy. I would suggest that it be named: "Energy Management of CO2 Capture Plant Under Market Uncertainties". I think the last "plant" is redundant and that one can infer that the article would talk about the most efficient method as most research aims to do. The "efficient" is redundant for that reason as well.   
\end{itemize}
{\bf{Special issue on efficient energy management}}

\begin{itemize}
\item This title is not effective nor indicative of the specific topic that is to be discussed within the energy management field. The scope is too broad and it is difficult for the reader to infer what issues or real-world applications might be discussed in the article. My suggestion would be to eradicate the "special issue" and replace with a more specific problem this article aims to focus on. 

\end{itemize}

{\bf{Mixed stochastic-deterministic tube MPC for offset-free tracking in the presence of plant-model mismatch }}

\begin{itemize}
\item This title is highly effective as well as specific enough to understand what the aforementioned "plant-model" will delve into. Not everyone may know what stochastic-deterministic means, but this is a positive thing as it encourages the reader to find out more about what the article might entail. This article title is not too wide but not so specific that the lengthiness hinders the reader from basic comprehension. 

\end{itemize}



\end{itemize}

\end{document}
