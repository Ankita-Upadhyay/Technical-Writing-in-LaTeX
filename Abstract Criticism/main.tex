\documentclass[11pt,twocolumn]{article}
\usepackage{fullpage}
\usepackage{amsmath}
\usepackage{spverbatim}
\usepackage{url}

\title{Abstract Criticism}
\author{Ankita Upadhyay}
\date{October 2018}

\begin{document}

\maketitle

\section{Paper One}

\begin{itemize}
  \item{\bf{Title}}: Bounds on the permanent and some applications 
  
  \item{\bf{Abstract}}:
We show that the permanent of a doubly stochastic $n \times n$ matrix $A = $\((a_{ij}\)) is at least as large as $\prod_{i,j} \((1-a_{ij}\))^{1-a_{ij}}$
and at most as large as $2^n$ times this number. Combined with previous
work, this improves on the deterministic approximation factor for the
permanent, giving $2^n$ instead of $e^n$-approximation.
We also give a combinatorial application of the lower bound, proving S. Friedland's "Asymptotic Lower
Matching Conjecture" for the monomer-dimer problem.

  \item{\bf{Violations}}:
  \begin{itemize}
  \item The title lacks specificity when it states "some applications".
  \item The title is not comprehensive and leaves too much room for interpretation.
  \item The abstract introduces mathematical equations and displays in-line formulas before providing an explanation or insight into the specific arithmetic topics it will cover. Ideally, an abstract should not even include any formulas.
  \item The abstract refers to previous work and is vague when it states "combined with previous work". Unless absolutely necessary, there should not be references to other articles.  
  \item The abstract doesn't indicate what the conclusions are. It merely lists what the article intends to prove without any references to the results.

  \end{itemize}
  
  \item{\bf{Revised Title}}: Bounds on the combinatorial permanent and stochastic applications
  
  \item{\bf{Revised Abstract}}: We show that the permanent of a doubly stochastic square matrix  is at least as large as the computable function of the doubly stochastic matrix and at most as large as $2^n$ times this number. A permanent is a linear algebra concept that denotes a polynomial entry in a given matrix. The mathematical proposition above improves on the deterministic approximation factor for the permanent, giving $2^n$ instead of $e^n$-approximation. We also give a combinatorial application of the lower bound, proving S. Friedland's "Asymptotic Lower Matching Conjecture" for the monomer-dimer problem. Then, we use combinatorial bounds on the permanent which results in a deterministic polynomial-time algorithm to approximate the permanent of a non negative matrix up to a multiplicative factor of $2^n$. \cite{bounds}  
  
\end{itemize}

\section{Paper Two}
\begin{itemize}
  \item{\bf{Title}}: Satisfiability and Evolution 
  \item{\bf{Abstract}}: 
We show that, if truth assignments on n variables reproduce through recombination so that satisfaction of a particular Boolean function confers a small evolutionary advantage, then a polynomially large population over polynomially many generations (polynomial in n and the inverse of the initial satisfaction probability) will end up almost certainly consisting exclusively of satisfying truth assignments. We argue that this theorem sheds light on the problem of the evolution of complex adaptations.

  \item{\bf{Violations}}:
  \begin{itemize}
  \item The title is obscure in nature and lacks specificity where it says "Satisfiability".  
  \item The word "Evolution" is vague and could refer to mathematics, biology, and/or other fields. It is not evident what the topic is that the title alludes to unless one reads the abstract. 
  \item The abstract does not provide a comprehensive summary of the content the paper entails. 
  \item The abstract does not include statements that suggest what the conclusions are.
  \item The abstract is too short as there are only two sentences. 
  \end{itemize}
  
  \item{\bf{Revised Title}}: Probability Satisfaction and Polynomially Deterministic Evolution
  
  \item{\bf{Revised Abstract}}: We show that, if truth assignments on n variables reproduce through recombination so that satisfaction of a particular Boolean function confers a small evolutionary advantage, then a polynomially large population over polynomially many generations (polynomial in n and the inverse of the initial satisfaction probability) will end up almost certainly consisting exclusively of satisfying truth assignments. We argue that this theorem sheds light on the problem of the evolution of complex adaptations. We first provide a background and model where we introduce the concept of product distribution and then proceed to discuss monotone functions until we share the proposed result. After examining six theorems and their respective proofs, we finally encounter the convergence proof which gauges the improvement in average population fitness obtained by the studies depicted. \cite{evolution} 
  
\end{itemize}

\bibliographystyle{unsrt}
\bibliography{main.bib}

\end{document}

